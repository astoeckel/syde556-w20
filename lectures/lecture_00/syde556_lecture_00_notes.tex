% !TeX spellcheck = en_GB
\documentclass[10pt,letterpaper,oneside]{article}
\usepackage{fontspec}
\usepackage{arev}
\usepackage[utf8]{inputenc}
\usepackage[T1]{fontenc}
\usepackage{amsmath}
\usepackage{amsfonts}
\usepackage{amssymb}
\usepackage{graphicx}
\usepackage{csquotes}
\usepackage{booktabs}
\usepackage{multicol}
\usepackage{enumerate}
\usepackage{microtype}
\usepackage[labelfont=bf,font={small}]{caption}
\usepackage{hyperref}
\usepackage{booktabs}
\usepackage{subcaption}
\usepackage{fancyhdr}
\usepackage{calc}
\usepackage[svgnames]{xcolor}
\usepackage{cleveref}

\newfontfamily\symbolfont{Symbola}
\usepackage[left=1in,right=1in,top=1in,bottom=1in,marginparwidth=0.25in]{geometry}

\usepackage[sorting=none]{biblatex}
\addbibresource{../bibliography.bib}

\author{Andreas Stöckel\\[0.5cm]Based on lecture notes by\\Chris Eliasmith and Terrence~C.~Stewart}

\fancyhf{}
\fancyhead[L]{SYDE 556/750 Lecture Notes}
\fancyhead[R]{Andreas Stöckel}
\fancyfoot[C]{\thepage}
\pagestyle{fancy}

\setlength{\parindent}{0em}
\setlength{\parskip}{0.5em}
\renewcommand{\baselinestretch}{1.25}
\renewcommand{\vec}[1]{{\mathbf{\mathrm{#1}}}}
\newcommand{\mat}[1]{{\mathbf{\mathrm{#1}}}}

\newcommand{\MakeTitle}[1]{
\maketitle
\begin{center}
	\includegraphics[width=0.5\textwidth]{../assets/uwlogo.pdf}\\[1cm]
	{#1}\
\end{center}

\vfill

\thispagestyle{empty}
\setcounter{page}{0}
\newpage

\pagenumbering{roman}
\tableofcontents
\newpage

\setcounter{page}{0}
\pagenumbering{arabic}}

\reversemarginpar

\newcommand{\ColorBox}[3]{{\par\hspace{0pt}\marginpar{\huge\raisebox{-1ex}{\symbolfont{#1}}}\ignorespaces\fboxsep=0.5cm\colorbox{#2}{\begin{minipage}[t]{\columnwidth-1cm}{#3}\end{minipage}}}}

\newcommand{\Note}[1]{\ColorBox{📌}{WhiteSmoke}{\textbf{Note:} #1}}
\newcommand{\Example}[1]{\ColorBox{💡}{WhiteSmoke}{\textbf{Example:} #1}}


\date{January 7, 2020}
\title{SYDE 556/750 \\ Simulating Neurobiological Systems \\ Lecture 0: Administrative Remarks}

\begin{document}

\MakeTitle{\textbf{Course website:}\\\url{http://compneuro.uwaterloo.ca/courses/syde-750.html}}

\section{Organization}

\begin{itemize}
	\item \textbf{Course website}\\
		  Links to all course material, including slides and these lecture notes and slides can be found at the following URLs:
		  \begin{itemize}
		  	\item \url{http://compneuro.uwaterloo.ca/courses/syde-750.html}
		  	\item \url{https://github.com/astoeckel/syde556-w20}
		  \end{itemize}
		  \emph{Note:} Any material on GitHub should be considered \enquote{preliminary} until officially linked at from the course website. Until then, the material is still subject to change.
	\item \textbf{Instructor}\\
		  Andreas Stöckel\\
		  Office: E7-6342 (office hours in E7-6323)\\
		  Email: \url{astoecke@uwaterloo.ca}\\
		  Website: \url{http://compneuro.uwaterloo.ca/people/andreas-stoeckel.html}
	\item \textbf{Course times and location}
		\begin{itemize}
			\item Tuesday: 11:30-12:50 in E5-4106 (SYDE 556/750)
		  	\item Thursday: 9:00-10:20 in E5-6004 (SYDE 556/750) 
		 	\item Thursday: 10:30-11:20 in E5-6127 (SYDE 750, optional for 556)
		\end{itemize}
	\item \textbf{Office hours}
		\begin{itemize}
			\item Office hours are generally in E7-6323 (this is a larger conference room).
			\item Time yet to be determined, one fixed office hour per week.
			\item Alternatively, if that time doesn't work for you, by appointment.
		\end{itemize}
	\item \textbf{Readings}
		\begin{itemize}
			\item Main resource: \enquote{Neural Engineering}, Chris Eliasmith and Charles Anderson, 2003 \cite{eliasmith2003neural}
			\item Optional: \enquote{How to Build a Brain}, Chris Eliasmith, 2012 \cite{eliasmith2013how}
		\end{itemize}
\end{itemize}

\newpage

\section{Coursework}

\begin{itemize}
	\item \textbf{Four assignments} (worth 60\% of the final mark)
	\begin{itemize}
		\item The assignments are worth 20\%, 20\%, 10\%, 10\% of the final mark, respectively.
		\item You have about two weeks for each assignment.
		\item You are free to discuss the assignments with other students, but do not take any (written) notes during such discussions. Everyone must write their own code, generate their own graphs, and write their own answers.
		\item These assignments (particularly the first two) are a lot of work, so start early.
	\end{itemize}
	\item \textbf{Final project} (worth 40\% of the final mark)
	\begin{itemize}
		\item We will have short project presentations in the final week, which are worth 5\% of the final mark. The written report will be worth 35\%.
		\item Build a model of some neural system
		\item For 556 students, this can be an extension of something seen in class or something that is listed in the book.
		\item For 750 students, this must be more of a research project with more novelty.
		\item Potential ideas are collected \href{http://compneuro.uwaterloo.ca/courses/syde-750/syde-556-possible-projects.html}{here}.
		\item In any case, your project idea needs to be approved via email before Reading Week (i.e., on February 14).
		\item See \href{http://compneuro.uwaterloo.ca/courses/syde-750/syde-556-possible-projects.html}{the project page} for more information.
	\end{itemize}
\end{itemize}

\section{Schedule}

\begin{center}
\small
\begin{tabular}{p{2.5cm} p{2.5cm} p{6cm} p{3.75cm}}
	\toprule
	\textbf{Date} &	\textbf{Reading} &	\textbf{Topic} & \textbf{Assignments} \\
	\midrule
	\footnotesize WEEK 1 & & & \\
	Jan 7 &
	Chapter 1 &
	Introduction &
	\\
	Jan 9 &
	Chapter 2 &
	Neurons &
	\\[0.125cm]
	
	\footnotesize WEEK 2 & & & \\
	Jan 14 &
	Chapter 2 &
	Population Representation (I) &
	\#1 posted\\
	Jan 16 &
	Chapter 2 &
	Population Representation (II) &
	\\[0.125cm]

	\footnotesize WEEK 3 & & & \\
	Jan 21 &
	Chapter 4 &
	Temporal Representation (I) &
	\\
	Jan 23 &
	Chapter 4 &
	Temporal Representation (II) &
	\\[0.125cm]

	\footnotesize WEEK 4 & & & \\
	Jan 28 &
	Chapters 5, 6 &
	Feedforward Transformations (I) &
	\#1 due*, \#2 posted\\
	Jan 30 &
	Chapters 5, 6 &
	Feedforward Transformations (II) &
	\\[0.125cm]

	\footnotesize WEEK 5 & & & \\
	Feb 4 &
	Chapter 8 &
	Dynamics (I) &
	\\
	Feb 6 &
	Chapter 8 &
	Dynamics (II) &
	\\[0.125cm]

	\footnotesize WEEK 6 & & & \\
	Feb 11 &
	Chapter 7 &
	Analysis of Representation &
	\#2 due*, \#3 posted\\
	Feb 13 &
	\emph{provided} &
	Temporal Basis Functions &
	\\
	Feb 14 &
	&
	&
	Project proposal due\\[0.125cm]

	\footnotesize WEEK 7 & \multicolumn{3}{c}{\emph{--- Reading week, no lectures ---}} \\[0.125cm]

	\footnotesize WEEK 8 & & & \\
	Feb 25 &
	\emph{provided} &
	Symbols (I) &
	\\
	Feb 27 &
	\emph{provided} &
	Symbols (II) &
	\\[0.125cm]

	\footnotesize WEEK 9 & & & \\
	Mar 3 &
	Chapter 8 &
	Memory &
	\#3 due*, \#4 posted\\
	Mar 5 &
	\emph{provided} &
	Action Selection &
	\\[0.125cm]

	\footnotesize WEEK 10 & & & \\
	Mar 10 &
	Chaper 9 &
	Learning (I) &
	\\
	Mar 12 &
	Chaper 9 &
	Learning (II) &
	\\[0.125cm]

	\footnotesize WEEK 11 & & & \\
	Mar 17 &
	\emph{provided} &
	Spatial Semantic Pointers &
	\#4 due*\\
	Mar 19 &
	\emph{provided} &
	Biological Details &
	\\[0.125cm]

	\footnotesize WEEK 12 & & & \\
	Mar 24 &
	\emph{provided} &
	Other modelling frameworks &
	\\
	Apr 2 &
	&
	Conclusion &
	\\[0.125cm]

	\footnotesize WEEK 13 & & & \\
	Mar 31, Apr 2 &
	&
	Project presentations &
	\\[0.125cm]

	\footnotesize WEEK 15 & & & \\
	Apr 15 &
	&
	&
	Projects due* \\
	\bottomrule
\end{tabular}\\[0.2cm]
\footnotesize
* The project and all assignments are due at midnight ($\approx$ 11:59p EST) of that day.
\end{center}

\newpage

\section{Things you should do to get started}

\begin{itemize}
	\item Get the textbook (\enquote{Neural Engineering}, Chris Eliasmith and Charles Anderson, 2003)
	\item Be able to run \texttt{jupyter lab} or \texttt{jupyter notebook} with a Python 3 kernel. Install \texttt{numpy}, \texttt{scipy}, and \texttt{matplotlib}. \href{https://www.anaconda.com/distribution/}{Anaconda} is a Python distribution that ships with these packets preinstalled, so (depending on your platform) this might be the easiest to use.
	\item Start thinking about a project\textellipsis already.
\end{itemize}

\printbibliography

\end{document}
