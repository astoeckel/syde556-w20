% !TeX spellcheck = en_GB
\documentclass[10pt,letterpaper,oneside]{article}
\usepackage{fontspec}
\usepackage{arev}
\usepackage[utf8]{inputenc}
\usepackage[T1]{fontenc}
\usepackage{amsmath}
\usepackage{amsfonts}
\usepackage{amssymb}
\usepackage{graphicx}
\usepackage{csquotes}
\usepackage{booktabs}
\usepackage{multicol}
\usepackage{enumerate}
\usepackage{microtype}
\usepackage[labelfont=bf,font={small}]{caption}
\usepackage{hyperref}
\usepackage{booktabs}
\usepackage{subcaption}
\usepackage{fancyhdr}
\usepackage{calc}
\usepackage[svgnames]{xcolor}
\usepackage{cleveref}

\newfontfamily\symbolfont{Symbola}
\usepackage[left=1in,right=1in,top=1in,bottom=1in,marginparwidth=0.25in]{geometry}

\usepackage[sorting=none]{biblatex}
\addbibresource{../bibliography.bib}

\author{Andreas Stöckel\\[0.5cm]Based on lecture notes by\\Chris Eliasmith and Terrence~C.~Stewart}

\fancyhf{}
\fancyhead[L]{SYDE 556/750 Lecture Notes}
\fancyhead[R]{Andreas Stöckel}
\fancyfoot[C]{\thepage}
\pagestyle{fancy}

\setlength{\parindent}{0em}
\setlength{\parskip}{0.5em}
\renewcommand{\baselinestretch}{1.25}
\renewcommand{\vec}[1]{{\mathbf{\mathrm{#1}}}}
\newcommand{\mat}[1]{{\mathbf{\mathrm{#1}}}}

\newcommand{\MakeTitle}[1]{
\maketitle
\begin{center}
	\includegraphics[width=0.5\textwidth]{../assets/uwlogo.pdf}\\[1cm]
	{#1}\
\end{center}

\vfill

\thispagestyle{empty}
\setcounter{page}{0}
\newpage

\pagenumbering{roman}
\tableofcontents
\newpage

\setcounter{page}{0}
\pagenumbering{arabic}}

\reversemarginpar

\newcommand{\ColorBox}[3]{{\par\hspace{0pt}\marginpar{\huge\raisebox{-1ex}{\symbolfont{#1}}}\ignorespaces\fboxsep=0.5cm\colorbox{#2}{\begin{minipage}[t]{\columnwidth-1cm}{#3}\end{minipage}}}}

\newcommand{\Note}[1]{\ColorBox{📌}{WhiteSmoke}{\textbf{Note:} #1}}
\newcommand{\Example}[1]{\ColorBox{💡}{WhiteSmoke}{\textbf{Example:} #1}}


\date{March 5, 2020}
\title{SYDE 556/750 \\ Simulating Neurobiological Systems \\ Lecture 9: Analysing Representation}


\begin{document}

\MakeTitle{\textbf{Accompanying Readings: Chapter 7 of Neural Engineering}}


\ConstructionSite

\begin{itemize}
	\item \textbf{Observation:} Some functions are \enquote{harder} to decoder than others (larger error)
	\item \textbf{Goal:} Get a better understanding of the types of function that can be decoded
	\item Tuning curves are a set of basis functions; decoders combine these basis functions
	\begin{align*}
		\hat x &= \sum_{i = 1}^n d_i a_i(x) = \langle \vec d, \vec a(x) \rangle
	\end{align*}
	\item Tuning curves are highly similar
	\item Find basis transformation $\mat T$ that maximises the information in the basis functions $\Rightarrow$ PCA
	\begin{align*}
		\hat x &= \langle \vec d, \mat T \vec a \rangle = \langle \vec d \mat T^{-1}, \mat T \vec a \rangle
	\end{align*}
	\item The scale Eigenvalues corresponding to the individual Principal Compnents is inversely proportional to the noise in the decoding $\Rightarrow$ large Eigenvalue $\Rightarrow$ this basis function can be decoded well
\end{itemize}

\printbibliography

\end{document}

